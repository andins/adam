The population activity, as recorded by multi-electrode array, can be regarded as a trajectory in a high-dimensional space.
In each trial the population traverses this space along a given trace.
Different retinal and real-world directions and velocities can produce population traces that are statistically different.
Because of connectivity patterns between recorded neurons
and common inputs the traces usually lye in a lower dimensional space.
We will apply manifold learning techniques to study the low-dimensional embeddings of population activity under the influence of
retinal and real-world direction and velocity of motion stimuli.
In particular linear methods are always the first choice for their simplicity and easy of estimation. Nonetheless a preliminary analysis
of MT data with linear PCA shows an interesting non-linear structure that neatly distinguish motion direction in a head-fixed setting (fig).
This result suggests that a non-linear technique could extract a more compact representation of data, thereby providing a clearer insight
into the underlying computational principles.
